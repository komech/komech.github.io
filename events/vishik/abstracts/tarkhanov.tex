\documentclass[10pt,a4paper]{article}

%% Title/abstract template for the conference in honour of Mark Vishik
%% The original is at http://www.dynamics.iitp.ru/vishik/abstract.tex
%%
%% Instructions:
%% 1. Title and abstract should be in English
%% 2. You are welcome to include the Russian translation (optional).
%%    Russian abstract could be shorter than English.
%% 3. Please make sure that your input, including References, is on one page

%% Enabling Cyrillic input with koi8-r encoding:
%\usepackage[T2A]{fontenc}
%\usepackage[koi8-r]{inputenc}
%\usepackage[russian,english]{babel}
%\usepackage[english]{babel}

%% Standard packages and definitions:

\usepackage{amssymb}
\usepackage{latexsym}
\usepackage{amsmath}

\def\R{\mathbb{R}}
\def\C{\mathbb{C}}
\def\Z{\mathbb{Z}}
\def\Q{\mathbb{Q}}
\def\p{\partial}

\pagestyle{empty}

\begin{document}

\begin{center}

%% Title in English:
{\Large Algebra of boundary value problems with small parameter}

\bigskip

%% Author 1:
{\sc Nikolai Tarkhanov}

%% Address:
{\small\it University of Potsdam, Germany}

%% Email:
{\small \rm tarkhanov@math.uni-potsdam.de}

%%\bigskip

%% Author 2:
%% {\sc Evgeniya Dyachenko}

%% {\small\it University of Potsdam}

%% {\small \rm dyachenk@uni-potsdam.de}

\end{center}

\bigskip

%% Abstract in English:

In a singular perturbation problem one is concerned with a differential equation of the form
$A (\varepsilon) u_\varepsilon = f_\varepsilon$
with initial or boundary conditions
$B (\varepsilon) u_\varepsilon = g_\varepsilon$,
where $\varepsilon$ is a small parameter.
The distinguishing feature of this problem is that the orders of
$A (\varepsilon)$ and
$B (\varepsilon)$
for $\varepsilon \neq 0$ are higher than the orders of $A (0)$ and
$B (0)$,
respectively.
%The differential problem in question is referred to as
%   a perturbed problem when $\varepsilon \neq 0$ and
%   a degenerate problem when $\varepsilon = 0$.
%The singular perturbation problem consists of studying the behaviour of %solutions or eigenvalues as $\varepsilon \to 0$.
%Such problems can also be considered with more than one parameter.
%
%Singular perturbation problems arise frequently in applied mathematics and have %been considered at least as far back in history as Lord Rayleigh's treatise
%\cite{Rayl45}, first published in 1877.
%Rayleigh considered the effect of a small amount of stiffness on the models of %vibration of a violin string.
%A discussion of the role of singular perturbation phenomena in mathematical %physics can be found in \cite{Frie55}.
%
%Some difficulties are inherent in singular perturbation problems.
%Solutions of the degenerate problem will not in general be as smooth as %solutions of the perturbed problem.
%Moreover, solutions of the degenerate problem usually will not satisfy as many %initial or boundary conditions as do solutions of the perturbed problem.
%Hence, if solutions of the perturbed problem are to converge to solutions of the %degenerate problem, the notion of convergence will probably have to be rather %weak.
%Due to the ``loss'' of initial or boundary data it may also happen that %solutions of the perturbed problem converge in a stronger sense in the interior %of the underlying domain, than in the vicinity of the boundary.
%This is known as the boundary layer phenomenon.
There is by now a vast amount of literature on singular perturbation
%problems for ordinary differential equations, both linear and non-linear.
%An extensive bibliography of this literature is contained in \cite{Waso66}.
%
%There is also a considerable amount of literature on singular perturbation
problems for partial differential equations.
A comprehensive theory of such problems was initiated by the remarkable paper of
Vishik and Lyusternik \cite{VishLyus57}.
%They obtained asymptotic expressions for solutions of the perturbed problem for %linear equations using boundary layer techniques.
%In this paper the main condition on the dependence of $A (\varepsilon)$ on a %small parameter was formulated and the asymptotics as $\varepsilon \to 0$ of the %solution of the Dirichlet problem was constructed.
%\cite{VishLyus57} also contains a sizable bibliography.
%
%In \cite{Huet60}, Huet published several theorems on convergence in singular %perturbation problems for linear elliptic and parabolic partial differential %equations.
%One particular feature distinguishes this paper from those previously mentioned.
%This is that convergence theorems are first proven in a Hilbert space setting %and then applied to the differential problems as opposed to starting directly %with the differential equations.
%In the elliptic vase, theorems on local convergence and convergence of %tangential derivatives at the boundary are also proven.
%The work \cite{Huet60} is fundamental to the considerations in \cite{Gree68} %aimed at obtaining rate of convergence estimates for solutions of singular %perturbations of linear elliptic boundary value problems.
%The problem can be described as follows.
%Let $\mathcal{X}$ be a compact smooth manifold and let $\varepsilon$ be a %positive real parameter.
%Consider two elliptic boundary value problems on $\mathcal{X}$,
%   $(\varepsilon \mathcal{A}_1) + \mathcal{A}_0) u_\varepsilon = f$ and
%   $\mathcal{A}_0 u = f$,
%where the order of $\mathcal{A}_1$ is greater than the order of $\mathcal{A}_0$.
%The problem is to determine in what sense $u_\varepsilon$ converges to $u$ on
%$\mathcal{X}$ as $\varepsilon \to 0$ and to estimate the rate of convergence.
%
%Pseudodifferential problems with small parameter were studied in the 1970s by
%G.~Eskin and A.~Demidov.
%For boundary value problems of general type the theory of singular perturbations %was developed in the 1980s by Frank, see \cite{Fran90}.
In \cite{Vole06}, Volevich completed the theory of differential boundary value problems with small parameter by formulating the Shapiro-Lopatinskii type ellipticity condition% and proving that it is equivalent to a priori estimates uniform in the parameter%
.

We contribute to the theory by constructing an algebra of pseudodifferential operators in which singularly perturbed boundary value problems can be treated.
Given any $m, \mu \in \R$, denote by
$\mathcal{S}^{m,\mu}$
the space of all smooth functions $a (x,\xi,\varepsilon)$ on
$T^\ast \R^n \times \R_{\geq 0}$,
such that
$
|D^\alpha_x D^\beta_\xi a|
\leq
C_{\alpha,\beta}\, <\xi>^{\mu-|\beta|} <\varepsilon \xi>^{m-\mu}
$
for all multi-indices $\alpha$ and $\beta$, where
$C_{\alpha,\beta}$ are constants independent of $x$, $\xi$ and $\varepsilon$.
For any fixed $\varepsilon > 0$, a function $a \in \mathcal{S}^{m,\mu}$ is a symbol of order $m$ on $\R^n$ which obviously degenerates as $\varepsilon \to 0$.
These symbols quantize to continuous operators
$H^{r,s} \to H^{r-m,s-\mu}$
in a scale of Sobolev spaces on $\R^n$ whose norms depend on $\varepsilon$ and are based on $L^2$ and weight functions
$<\xi>^{s} <\varepsilon \xi>^{r-s}$.
The family $\mathcal{S}^{m-j,\mu-j}$ with $j = 0, 1, \ldots$ is used as usual to define asymptotic sums of homogeneous symbols.
By the homogeneity of degree $\mu$ is meant the property
$
a (x, \lambda \xi, \lambda^{-1} \varepsilon)
= \lambda^\mu a (x, \xi, \varepsilon)
$
for all $\lambda > 0$.
Let $\mathcal{S}^{m,\mu}_{\mathrm{phg}}$ stand for the subspace of
$\mathcal{S}^{m,\mu}$
consisting of all polyhomogeneous symbols, i.e., those admitting asymptotic expansions in homogeneous symbols.
For any $a \in \mathcal{S}^{m,\mu}_{\mathrm{phg}}$ there is well-defined principal homogeneous symbol $\sigma^\mu (a)$ of degree $\mu$ whose invertibility away from the zero section of $T^\ast \R^n$ is said to be the interior ellipticity with small parameter.
%
Familiar techniques lead now to calculi of pseudodifferential operators with small parameter on diverse compactifications of smooth manifolds.
Our results gain in interest if we realize that pseudodifferential operators with small parameter provide also adequate tools for studying Cauchy problems for elliptic equations.

This is a joint paper with my PhD student Evgeniya Dyachenko who studies singular perturbation problems.

%% If you use BIBTEX to create bibliography, please use amsalpha:
%\bibliographystyle{amsalpha}
%\bibliography{physics}%% (if your BIBTEX entries are in physics.bib)
%\end{document}

\begin{thebibliography}{5}
%
%\bibitem[AV64]{AgraVish64}
%M.~S. Agranovich and M.~I. Vishik,
%  \emph{Elliptic problems with a parameter and parabolic problems of general
%        type},
%  Uspekhi Mat. Nauk \textbf{19} (1964), Issue 3, 53--161.
%
%\bibitem[BdM71]{Bout71}
%L. Boutet de Monvel,
%  \emph{Boundary problems for pseudo-differential operators},
%  Acta Math. \textbf{126} (1971), no.~1--2, 11--51.
%
%\bibitem[Dem75]{Demi75}
%A.~S. Demidov,
%  \emph{Asymptotic behaviour of the solution of a boundary value problem for
%        elliptic pseudodifferential equations with a small parameter multiplying
%        the highest operator},
%  Trans. Moscow Math. Soc. \textbf{32} (1975), 119--146.
%
%\bibitem[Fra90]{Fran90}
%L. S. Frank,
%  \emph{Spaces and Singular Perturbations on Manifolds without Boundary},
%  North Holland, Amsterdam, 1990.
%
%\bibitem[Fri55]{Frie55}
%K. Friedrichs,
%  \emph{Asymptotic phenomena in mathematical physics},
%  Bull. Amer. Math. Soc. \textbf{61} (1955), 485--504.
%
%\bibitem[Gre68]{Gree68}
%W. M. Greenlee,
%  \emph{Rate of convergence in singular perturbations},
%   Ann. Inst. Fourier (Grenoble) {\bf 18} (1968), no. 2, 135--191.
%
%\bibitem[Hue60]{Huet60}
%D. Huet,
%  \emph{Ph\'{e}nom\`{e}nes de perturbation singuli\`{e}re dans les probl\`{e}mes
%        aux limites},
%   Ann. Inst. Fourier (Grenoble) {\bf 11} (1961), 385--475.
%
%\bibitem[Ray45]{Rayl45}
%Lord Rayleigh,
%  \emph{Theory of Sound},
%  Vol. I and II, Dover, 1945.
%
\bibitem{VishLyus57}
M.~I. Vishik and L.~A. Lyusternik,
\emph{Regular degeneration and boundary layer for linear differential equations
with small parameter},
Uspekhi Mat. Nauk \textbf{12} (1957), Issue 5 (77), 3--122.

\bibitem{Vole06}
L.~R. Volevich,
\emph{The Vishik-Lyusternik method in elliptic problems with small parameter},
Trans. Moscow Math. Soc. \textbf{67} (2006), 87--125.
%
%\bibitem[Was66]{Waso66}
%W. Wasow,
%  \emph{Asymptotic expansions for ordinary differential equations},
%  Wiley, 1966.
%
\end{thebibliography}

\end{document}
