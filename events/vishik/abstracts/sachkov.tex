\documentclass[10pt,a4paper]{article}

%% Title/abstract template for the conference in honour of Mark Vishik
%% The original is at http://www.dynamics.iitp.ru/vishik/abstract.tex
%%
%% Instructions:
%% 1. Title and abstract should be in English
%% 2. You are welcome to include the Russian translation (optional).
%%    Russian abstract could be shorter than English.
%% 3. Please make sure that your input, including References, is on one page

%% Enabling Cyrillic input with koi8-r encoding:
%\usepackage[T2A]{fontenc}
%\usepackage[koi8-r]{inputenc}
%\usepackage[russian,english]{babel}
\usepackage[english]{babel}

%% Standard packages and definitions:
\usepackage{amssymb}
\usepackage{latexsym}
\usepackage{amsmath}
\def\R{\mathbb{R}}
\def\C{\mathbb{C}}
\def\Z{\mathbb{Z}}
\def\Q{\mathbb{Q}}
\def\p{\partial}
\pagestyle{empty}

\begin{document}
\begin{center}

%% Title in English:
{\Large  Neurogeometry of vision and sub-Riemannian geometry}

\bigskip

%% Author 1:
{\sc Yuri Sachkov}

%% Address:
{\small\it Program systems institute, Pereslavl-Zalessky 152020,  Russia}

%% Email:
{\small\rm sachkov@sys.botik.ru}


\end{center}

\bigskip

%% Abstract in English:
The talk will be devoted to the following questions:
\begin{itemize}
\item 
Image inpainting
\item 
The pinwheel model of the primary visual cortex V1 of a human brain,
\item 
Sub-Riemannian problem on the group of rototranslations of a plane  and its solution,
\item 
Image inpainting via sub-Riemannian length minimizers,
\item 
Curve cuspless reconstruction,
\item 
Image inpainting via hypoelliptic diffusion.
\end{itemize}


\bigskip


%% If you use BIBTEX to create bibliography, please use amsalpha:
%\bibliographystyle{amsalpha}
%\bibliography{physics}%% (if your BIBTEX entries are in physics.bib)
%\end{document}

\begin{thebibliography}{Sch26}

\bibitem[P1]{petitot}
J.Petitot,
 \emph{The neurogeometry
	of pinwheels as a sub-Riemannian contact structure},
J. Physiology - Paris,    \textbf{97} (2003), 265--309.

\bibitem[P2]{petitot2}
J.Petitot,
 \emph{Neurogeometrie de la vision --- Modeles mathematiques et physiques des architectures fonctionnelles},
  (2008), Editions de l'Ecole Polytechnique. 


\bibitem[S1]{max_sre}
Yuri L. Sachkov and Igor Moiseev,
 \emph{Maxwell strata in sub-Riemannian problem  on the group of motions of a plane},
 ESAIM: COCV,    \textbf{16} (2010), 380--399.

\bibitem[S2]{cut_sre1}
Yuri L. Sachkov,
 \emph{Conjugate and cut time in the sub-Riemannian problem on the group of motions of a 
plane},
 ESAIM: COCV,    \textbf{16} (2010), 1018--1039.
 
\bibitem[S3]{cut_sre2}
Yuri L. Sachkov,
 \emph{Cut locus and optimal synthesis in the sub-Riemannian problem  on the group of motions of a 
plane},
 ESAIM: COCV,    \textbf{17} (2011), 293--321.
 

\end{thebibliography}

\end{document}
