\documentclass[10pt,a4paper]{article}

%% Title/abstract template for the conference in honour of Mark Vishik
%% The original is at http://www.dynamics.iitp.ru/vishik/abstract.tex
%%
%% Instructions:
%% 1. Title and abstract should be in English
%% 2. You are welcome to include the Russian translation (optional).
%%    Russian abstract could be shorter than English.
%% 3. Please make sure that your input, including References, is on one page

%% Enabling Cyrillic input with koi8-r encoding:
%\usepackage[T2A]{fontenc}
%\usepackage[koi8-r]{inputenc}
%\usepackage[russian,english]{babel}
%\usepackage[english]{babel}

%% Standard packages and definitions:
\usepackage{amssymb}
\usepackage{latexsym}
\usepackage{amsmath}
\def\R{\mathbb{R}}
\def\C{\mathbb{C}}
\def\Z{\mathbb{Z}}
\def\Q{\mathbb{Q}}
\def\p{\partial}
\pagestyle{empty}

\begin{document}
\begin{center}

%% Title in English:
{\Large Relative version of the Titchmarsh convolution theorem}

\bigskip

%% Author 1:
{\sc Evgeny Gorin}

%% Address:
{\small\it Moscow State Pedagogical University, Moscow, Russia}

%% Email:
{\small\rm evgeny.gorin@mtu-net.ru}

\bigskip

{\sc Dmitry Treschev}

{\small\it Steklov Mathematical Institute}

{\small\rm treschev@mi.ras.ru}

\end{center}

\bigskip


We consider the algebra
$C_u = C_u (\R)$
of uniformly continuous bounded complex functions on the real line $\R$
with pointwise operations and sup-norm. Let $I$ be a closed ideal in $C_u$
invariant with respect to translations, and let $\mathrm{ah}_I (f )$
denote the minimal real number (if it exists)
satisfying the following condition. If
$\lambda > \mathrm{ah}_I (f )$, then $(\hat f - \hat g )|_V = 0$
for some $g_I$ , where $V$ is
a neighborhood of the point $\lambda$.
The classical Titchmarsh convolution theorem is equivalent to the
equality
$\mathrm{ah}_I (f_1 \cdot f_2 ) = \mathrm{ah}_I (f_1 ) + \mathrm{ah}_I (f_2 )$,
where $I = \{0\}$.
We show that, for ideals $I$ of general
form, this equality does not generally hold,
but $\mathrm{ah}_I (f^n ) = n \cdot \mathrm{ah}_I (f )$ holds for any $I$.
We present
many nontrivial ideals for which the general form
of the Titchmarsh theorem is true.


\end{document}
