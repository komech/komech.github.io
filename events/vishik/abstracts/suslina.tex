\documentclass[10pt,a4paper]{article}

%% Title/abstract template for the conference in honour of Mark Vishik
%% The original is at http://www.dynamics.iitp.ru/vishik/abstract.tex
%%
%% Instructions:
%% 1. Title and abstract should be in English
%% 2. You are welcome to include the Russian translation (optional).
%%    Russian abstract could be shorter than English.
%% 3. Please make sure that your input, including References, is on one page

%% Enabling Cyrillic input with koi8-r encoding:
%\usepackage[T2A]{fontenc}
%\usepackage[koi8-r]{inputenc}
%\usepackage[russian,english]{babel}
%\usepackage[english]{babel}

%% Standard packages and definitions:
\usepackage{amssymb}
\usepackage{latexsym}
\usepackage{amsmath}
\def\R{\mathbb{R}}
\def\C{\mathbb{C}}
\def\Z{\mathbb{Z}}
\def\Q{\mathbb{Q}}
\def\p{\partial}
\pagestyle{empty}

\begin{document}
\begin{center}

%% Title in English:
{\Large Homogenization of the elliptic Dirichlet problem: operator error estimates}

\bigskip

%% Author 1:
{\sc T.A.~Suslina}

%% Address:
{\small\it St.~Petersburg State University, St.~Petersburg, Russia}

%% Email:
{\small\rm suslina@list.ru}

\end{center}

\bigskip

%% Abstract in English:
Let $\mathcal{O} \subset \mathbb{R}^d$ be a bounded domain of class $C^{1,1}$.
In $L_2(\mathcal{O};\mathbb{C}^n)$, we consider a matrix elliptic differential
operator $A_{\varepsilon}= b(\mathbf{D})^* g(\mathbf{x}/\varepsilon) b(\mathbf{D})$
with the Dirichlet boundary condition. We assume that
an $(m\times m)$-matrix-valued function $g(\mathbf{x})$
is bounded, uniformly positive definite and periodic
with respect to some lattice $\Gamma$. The elementary cell of $\Gamma$ is denoted by $\Omega$.
Next, $b(\mathbf{D})=\sum_{j=1}^d b_j D_j$ is an $(m\times n)$-matrix first
order differential operator ($b_j$ are constant
matrices). It is assumed that $m \ge n$ and the symbol $b(\boldsymbol{\xi})= \sum_{j=1}^d b_j \xi_j$
has maximal rank, i.~e., $\text{rank}\,b(\boldsymbol{\xi})= n$ for $0 \ne \boldsymbol{\xi} \in
\mathbb{R}^d$. The simplest example is $A_{\varepsilon} = -\text{div}\,g(\mathbf{x}/\varepsilon)\nabla$.

We study the behavior of the solution $\mathbf{u}_\varepsilon$ of the Dirichlet problem
$A_{\varepsilon}\mathbf{u}_\varepsilon = \mathbf{F}$ in $\mathcal{O}$,
$\mathbf{u}_\varepsilon\vert_{\partial \mathcal{O}} =0$,
where $\mathbf{F} \in L_2(\mathcal{O};\mathbb{C}^n)$.
It turns out that $\mathbf{u}_\varepsilon$ converges in $L_2(\mathcal{O};\mathbb{C}^n)$
to $\mathbf{u}_0$, as $\varepsilon \to 0$. Here $\mathbf{u}_0$ is the solution of the "homogenized"\
Dirichlet problem $A^0 \mathbf{u}_0=\mathbf{F}$ in $\mathcal{O}$,
$\mathbf{u}_0\vert_{\partial \mathcal{O}} =0$. The \textit{effective operator}
$A^0$ is given by the expression $A^0 = b(\mathbf{D})^* g^0 b(\mathbf{D})$
with the Dirichlet boundary condition. The effective matrix $g^0$ is a constant positive $(m\times m)$-matrix
defined as follows. Denote by $\Lambda(\mathbf{x})$ the $(n \times m)$-matrix-valued
periodic solution of the equation
$b(\mathbf{D})^* g(\mathbf{x}) (b(\mathbf{D})\Lambda(\mathbf{x}) + \mathbf{1}_m) =0$
such that $\int_\Omega \Lambda(\mathbf{x})\, d\mathbf{x} =0$.
Then $g^0 = |\Omega|^{-1} \int_\Omega g(\mathbf{x}) (b(\mathbf{D})\Lambda(\mathbf{x})
+ \mathbf{1}_m)\,d\mathbf{x}$.

\smallskip
\noindent\textbf{Theorem 1.} (see [2]) \textit{We have the following sharp order
error estimate}:
$$
\| \mathbf{u}_\varepsilon - \mathbf{u}_0 \|_{L_2(\mathcal{O};\mathbb{C}^n)}
\le C \varepsilon \| \mathbf{F}\|_{L_2(\mathcal{O};\mathbb{C}^n)}.
$$


Now we give approximation of $\mathbf{u}_\varepsilon$ in the Sobolev space
$H^1(\mathcal{O};\mathbb{C}^n)$. For this, the first order corrector must be taken into account.

\smallskip
\noindent\textbf{Theorem 2.} (see [1]) 1) \textit{Let $\Lambda \in L_\infty$, and denote
$\Lambda^\varepsilon(\mathbf{x}) = \Lambda(\varepsilon^{-1} \mathbf{x})$. Then}
$$
\| \mathbf{u}_\varepsilon - \mathbf{u}_0 - \varepsilon \Lambda^\varepsilon b(\mathbf{D})
\mathbf{u}_0 \|_{H^1(\mathcal{O};\mathbb{C}^n)}
\le C \varepsilon^{1/2} \| \mathbf{F}\|_{L_2(\mathcal{O};\mathbb{C}^n)}.
$$

\noindent 2) \textit{In the general case, we have}
$$
\| \mathbf{u}_\varepsilon - \mathbf{u}_0 - \varepsilon \Lambda^\varepsilon b(\mathbf{D})
(S_\varepsilon \widetilde{\mathbf{u}}_0) \|_{H^1(\mathcal{O};\mathbb{C}^n)}
\le C \varepsilon^{1/2} \| \mathbf{F}\|_{L_2(\mathcal{O};\mathbb{C}^n)}.
$$
\textit{Here $\widetilde{\mathbf{u}}_0 = P_{\mathcal{O}} \mathbf{u}_0$ and
$P_{\mathcal{O}}: H^2(\mathcal{O};\mathbb{C}^n) \to H^2(\mathbb{R}^d;\mathbb{C}^n)$
is a continuous extension operator, $S_\varepsilon$ is the smoothing operator
$(S_\varepsilon \mathbf{u})(\mathbf{x}) = |\Omega|^{-1} \int_\Omega
\mathbf{u}(\mathbf{x} - \varepsilon \mathbf{z})\,d\mathbf{z}$.}

%\smallskip
%We use the results of M.~Birman and T.~Suslina for homogenization problem in $\mathbb{R}^d$:
%the analogs of estimates (2), (3) in $\mathbb{R}^d$ are of sharp order $\varepsilon$.
%The problem is reduced to estimating of the discrepancy $\mathbf{w}_\varepsilon$,
%which is the solution of the problem $A_\varepsilon \mathbf{w}_\varepsilon =0$ in $\mathcal{O}$,
%$\mathbf{w}_\varepsilon\vert_{\partial \mathcal{O}} =
%\varepsilon \Lambda^\varepsilon b(\mathbf{D})
%(S_\varepsilon \widetilde{\mathbf{u}}_0)\vert_{\partial \mathcal{O}}$
%(or $\mathbf{w}_\varepsilon\vert_{\partial \mathcal{O}} =
%\varepsilon \Lambda^\varepsilon b(\mathbf{D}){\mathbf{u}}_0\vert_{\partial \mathcal{O}}$
%in the case $\Lambda \in L_\infty$).
%We show that the norm of $\mathbf{w}_\varepsilon$ in $H^1$ satisfies estimate
%of order $\varepsilon^{1/2}$. This leads to (2), (3).
%At the same time, the norm of $\mathbf{w}_\varepsilon$ in $L_2$ is of order $\varepsilon$,
%this allows us to prove sharp order estimate (1).


\begin{thebibliography}{6}

\bibitem{pa-su-2012} M.~A.~Pakhnin, T.~A.~Suslina, \textit{Operator error estimates for homogenization of the
Dirichlet problem in a bounded domain}, Preprint, 2012. Available at
http://arxiv.org/abs/1201.2140.


\bibitem{su-2012} T.~A.~Suslina, \textit{Homogenization of the elliptic Dirichlet problem:
operator error estimates in $L_2$}, Preprint, 2012. Available at http://arxiv.org/abs/1201.2286.




\end{thebibliography}

\end{document}
