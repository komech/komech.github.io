\documentclass[10pt,a4paper]{article}

%% Title/abstract template for the conference in honour of Mark Vishik
%% The original is at http://www.dynamics.iitp.ru/vishik/abstract.tex
%%
%% Instructions:
%% 1. Title and abstract should be in English
%% 2. You are welcome to include the Russian translation (optional).
%%    Russian abstract could be shorter than English.
%% 3. Please make sure that your input, including References, is on one page

%% Enabling Cyrillic input with koi8-r encoding:
\usepackage[T2A]{fontenc}
%\usepackage[koi8-r]{inputenc}
\usepackage[english]{babel}

%% Standard packages and definitions:
\usepackage{amssymb}
\usepackage{latexsym}
\usepackage{amsmath}
\def\R{\mathbb{R}}
\def\C{\mathbb{C}}
\def\Z{\mathbb{Z}}
\def\Q{\mathbb{Q}}
\pagestyle{empty}

\begin{document}
\begin{center}

%% Title in English:
{\Large Vishik--Lyusternik's method and the inverse problem for plasma equilibrium in a~tokamak}
\bigskip

%% Author 1:
{\sc Alexandre Demidov}

%% Address:
{\small\it Moscow State University, Moscow 119992, Russia}

%% Email:
{\small\rm alexandre.demidov@mtu-net.ru}

\end{center}

\bigskip

%% Abstract in English:

Control over thermonuclear fusion reactions
(including suppression of instabilities of the plasma discharge)
depends essentially on how well the information about the current
density through plasma is taken into account. In the case
cylindrical approximation (when the tokamak (toroidal magnetic)
chamber and the resulting plasma discharge are modeled in the form
of infinite cylinders ${\mathcal S}\times\R$ and~$\omega\times\R$
with simply connected cross-sections ${\mathcal S}\Subset\R^2$
and~$\omega\Subset{\mathcal S}$), the required current
distribution is given by the mapping
$
f_{u}:\omega\ni (x,y)\mapsto f\bigl(u(x,y)\bigl)\ge 0,
$
where the \textit{required} functions $u\in C^2(\omega)$ and~$f$
are as follows:
\begin{equation*}\label{0.2}
\frac{\partial^2 u(x,y)}{\partial x^2}+\frac{\partial^2 u(x,y)}{\partial y^2}=f\bigl(u(x,y)\bigl)
\quad \ {\rm in}\quad \omega\,,\quad \ {\rm and}\quad \ u=0\quad {\rm on}\quad
\gamma=\partial\omega,
\end{equation*}
\begin{equation*}\label{0.3}
\sup\limits_{P\in\gamma}\Bigl|\frac{\partial
u}{\partial\nu}(P)-~\Phi(P)\Bigl|\le\lambda\sup\limits_{P\in\gamma}\Bigl|\Phi(P)\Bigl|
\,,\qquad \int_{\gamma}\frac{\partial u}{\partial \nu}\,d\gamma=1
=\mbox{the total current}\,.
\end{equation*}
Here, $\gamma=\overline{\omega}\setminus\omega$ is the boundary of
the domain $\omega$, $\lambda\ge 0$ is small parameter, $\nu$ is
the outward  unit normal to the curve~$\gamma=\partial\omega$
(with respect to the domain~$\omega)$. Both the function~$\Phi$
and the curve $\gamma=\partial\omega$ (and hence the
domain~$\omega$) may be regarded as known: they are determinable
from measurements of the magnetic field at the tokamak chamber
$\partial {\mathcal S}.$

Within the class of affine functions~$f: u\mapsto f(u)=au+b$,
Vishik--\allowbreak Lyusternik's method is capable of showing that
\begin{equation*}\label{uasymp}
\left|\frac{{\partial } u}{{\partial
}\nu}\biggl|_{s\in\gamma}-\left(\frac1{|\gamma|}-
\frac{k(s)-|\gamma|^{-1}\int_\gamma k(s)\,ds}{2|\gamma|\sqrt{a}}\right)\right|\le
\frac{C_{\gamma}(a)}{\sqrt{a}}\,,
\ C_{\gamma}(a)\to 0\;\; \text{as}\;\;
a\to\infty\,,
\end{equation*}
\begin{equation*}\label{Vasump}
\left|\frac{d}{d a}\frac{{\partial } u}{{\partial
}\nu}\biggl|_{s\in\gamma}-\; \frac{k(s)-|\gamma|^{-1}\int_\gamma
k(s)\,ds}{4|\gamma|\,{a}^{3/2}}\right|\le
\frac{C_{\gamma}(a)}{a^{3/2}}\,,\quad C_{\gamma}(a)\to 0\quad
\text{as}\quad
a\to\infty\,,
\end{equation*}
where $k(s)$ is the curvature of~$\gamma$ at~$s\in\gamma$. Using
these asymptotic relations, it follows that there is only one
affine distribution $f_u$ (for a~large class of domains~$\omega$)
if $\lambda\overset{}{=}0$. However, for any arbitrarily small
$\lambda>0$, there is an infinite number
$\{f^j_u\}_{j\in\mathbb{N}}$ of distributions, for which
$\|f^{j_1}_u\|\ll\|f^{j_2}_u\|$, $j_1\ne j_2$, where $\|f^j_u\|=
\max\limits_{(x,y)\in\omega}\left|f^j_u(x,y)\right|\,.$
It is shown that all these different distributions are necessarily
members of a~sequence converging to the $\delta$-function
supported on~$\gamma$ (the so-called skinned current). Hence these
distributions are not essentially different from the physical
standpoint.

Two truly physically essentially different current distributions
$f^1_{u}$ and $f^2_{u}$ are found in the class of polynomials~$f:
u\mapsto f(u)=\sum_{m=0}^3a_mu^m$ of third degree (see
Russian J.~Math.\ Physics, {\bf 17}~(1), 56--65 (2010) and
Asymptotic Analysis, {\bf 74}~(1), 95--121  (2011)).



\end{document}
