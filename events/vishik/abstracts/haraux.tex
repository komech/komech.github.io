\documentclass[10pt,a4paper]{article}

%% Title/abstract template for the conference in honour of Mark Vishik
%% The original is at http://www.dynamics.iitp.ru/vishik/abstract.tex
%%
%% Instructions:
%% 1. Title and abstract should be in English
%% 2. You are welcome to include the Russian translation (optional).
%%    Russian abstract could be shorter than English.
%% 3. Please make sure that your input, including References, is on one page

%% Enabling Cyrillic input with koi8-r encoding:

%% Standard packages and definitions:
\usepackage[T1]{fontenc}
\usepackage[latin1]{inputenc}
\usepackage[english]{babel}

\usepackage{amssymb}
\usepackage{latexsym}
\usepackage{amsmath}
\def\R{\mathbb{R}}
\def\C{\mathbb{C}}
\def\Z{\mathbb{Z}}
\def\Q{\mathbb{Q}}
\pagestyle{empty}

\begin{document}
\begin{center}

%% Title in English:
{\Large On a compactness problem}

\bigskip

%% Author 1:
{\sc Alain Haraux}

%% Address:
{\small\it Universit{\'e} Pierre et Marie Curie, Paris, France}

%% Email:
{\small\rm haraux@ann.jussieu.fr}




\end{center}

\bigskip

%% Abstract in English:

Let $V,  H $  be two real Hilbert spaces,  $V\subset H$ with compact and dense imbedding and let $A: V\rightarrow V' $ be bounded, self ajoint and coercive.  Under reasonable assumptions on $g$ , a  compactness property of the range is established for energy-bounded solutions of an abstract equation 
$$ u'' + Au + g(u') = h(t), \quad t\ge 0$$ when $h$ is $S^1$-uniformly continuous with values in $H$. This property allows to deduce\,:\smallskip

\noindent
1) asymptotic almost-periodicity of all solutions of wave or plate equations in presence of an almost-periodic source term when the damping term is strong enough.
\smallskip

\noindent
2) convergence to equilibrium of all solutions of some equations of the form 
$$ u'' + Au + f(u) + g(u') = h(t), \quad t\ge 0$$ when f is the gradient of a potential satisfying the {\L}ojasiewicz gradient inequality, $g$ is sufficiently coercive globally and $h(t) \rightarrow 0$  sufficiently fast for $t$ tending to infinity. 

\smallskip

These results generalize, mainly to the case of non-local damping terms,  some previous works by the author and his colleagues and rely on methods developped during more than 30 years, cf. e.g. \cite{haraux1, haraux2, haraux3} for applications of type 1) and \cite{haraux4,haraux5} for applications of type 2). There are still challenging open problems in this direction which will be mentionned during the lecture.


\begin{thebibliography}{7}
\bibitem{haraux1} L. Amerio \& G. Prouse, \emph{Uniqueness and almost-periodicity theorems for a non linear wave equation}, Atti Accad. Naz. Lincei Rend. Cl. Sci. Fis. Mat. Natur. \textbf{8}, 46 (1969) 1--8.

\bibitem{haraux2} M. Biroli, \& A. Haraux, \emph{Asymptotic behavior for an almost periodic, strongly dissipative wave equation}. J. Differential Equations \textbf{38}  (1980), no. 3, 422--440.

\bibitem{haraux3}A. Haraux, \emph{Damping out of transient states for some semi-linear, quasi-autonomous systems
of hyperbolic type}, Rc. Accad. Naz. Sci. dei 40 (Memorie di Matematica)  \textbf{101} 7, fasc.7 (1983), 89--136.

\bibitem{haraux4}A. Haraux \& M.A. Jendoubi, \emph{Convergence of bounded weak solutions of the wave equation with dissipation and analytic nonlinearity.}
Calc. Var. Partial Differential Equations \textbf{9} (1999), no. 2, 95--124.

\bibitem{haraux5}
I. Ben Hassen \& L.~Chergui, \emph{Convergence of global and bounded solutions of some nonautonomous second order evolution equations with nonlinear dissipation},  J. Dynam. Differential Equations \textbf{23}(2011), no. 2, 315--332.


\end{thebibliography}

\end{document}
