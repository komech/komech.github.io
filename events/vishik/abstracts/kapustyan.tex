\documentclass[10pt,a4paper]{article}

%% Title/abstract template for the conference in honour of Mark Vishik
%% The original is at http://www.dynamics.iitp.ru/vishik/abstract.tex
%%
%% Instructions:
%% 1. Title and abstract should be in English
%% 2. You are welcome to include the Russian translation (optional).
%%    Russian abstract could be shorter than English.
%% 3. Please make sure that your input, including References, is on one page

%% Enabling Cyrillic input with koi8-r encoding:
\usepackage[T2A]{fontenc}
\usepackage[koi8-r]{inputenc}
\usepackage[russian,english]{babel}

%% Standard packages and definitions:
\usepackage{amssymb}
\usepackage{latexsym}
\usepackage{amsmath}
\def\R{\mathbb{R}}
\def\C{\mathbb{C}}
\def\Z{\mathbb{Z}}
\def\Q{\mathbb{Q}}
\pagestyle{empty}

\begin{document}
\begin{center}

%% Title in English:
{\Large Structure and regularity of the global attractor of reaction-diffusion equation with non-smooth nonlinear term}

\bigskip

%% Author 1:
{\sc Aleksey Kapustyan}

%% Address:
{\small\it Kyiv National Taras Shevchenko University, Kyiv, Ukraine}

%% Email:
{\small\rm alexkap@univ.kiev.ua}


\bigskip

%% Author 2:
{\sc Pavel Kasyanov}

{\small\it  National Technical University of Ukraine, Kyiv, Ukraine}

{\small\rm kasyanov@i.ua}

\bigskip

%% Author 3:
{\sc Jose Valero}

{\small\it Universidad Miguel Hernandez de Elche,  Elche, Spain}

{\small\rm jvalero@umh.es}

\end{center}

\bigskip

%% Abstract in English:

In a bounded domain $\Omega\subset\mathbb{R}^{3}$ with sufficiently
smooth boundary $\partial\Omega$ we consider the problem
\begin{equation}
\left\{
\begin{array}
[c]{l}%
u_{t}-\Delta u+f(u)=h,\quad x\in\Omega,\ t>0,\\
u|_{\partial\Omega}=0,
\end{array}
\right. \label{ka1}%
\end{equation}
where $f\in C(\mathbb{R})$ satisfies suitable growth and dissipative
conditions, but there is no condition ensuring uniqueness of the
Cauchy problem. When the nonlinear term $f$ is smooth and $f'$
satisfies additional assumptions, it is well known
\cite{BabinVishik1989}, that the problem (\ref{ka1}) generates semigroup,
which has global attractor and it coincides with the unstable set,
emanating from the set of stationary points and with stable one as
well. In general case (2), when the uniqueness of Cauchy problem is
not guaranteed, we have existence of trajectory attractor
\cite{ChepVishik2002}, and existence of global attractor of
multivalued semiflow \cite{KapValero2010}. Our aim is to study the
structure of the global attractor in multi-valued case. We prove
that the attractor of the multi-valued semiflow generated by all
weak solutions of (\ref{ka1}) in the phase space $L^{2}\left( \Omega\right)
$ is the closure of the union of all stable manifolds of the set of
stationary points. Also,
for multi-valued semiflow,
generated by regular solutions,  we prove the existence of  global
attractor,  which is compact in $H_{0}^{1}\left( \Omega\right) $ and
we establish that it is the union of all unstable manifolds of the
set of stationary points and of the stable ones as well.


\begin{thebibliography}{6}

\bibitem{BabinVishik1989} A.V. Babin and M.I. Vishik, \emph{Attractors of evolution
equations}, Nauka, Moscow, 1989.

\bibitem{ChepVishik2002} V.V. Chepyzhov and M.I. Vishik,
\emph{Attractors for equations of mathematical physics},
AMS, Providence, 2002.

\bibitem{KapValero2010} O.V. Kapustyan and J. Valero,
\emph{Comparison between trajectory and global attractors for evolution systems without uniqueness of solutions}, Int. J. Bif. and Chaos.
\textbf{20} (2010), 2723--2734.

\end{thebibliography}

\end{document}
