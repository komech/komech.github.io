\documentclass[10pt,a4paper]{article}

%% Title/abstract template for the conference in honour of Mark Vishik
%% The original is at http://www.dynamics.iitp.ru/vishik/abstract.tex
%%
%% Instructions:
%% 1. Title and abstract should be in English
%% 2. You are welcome to include the Russian translation (optional).
%%    Russian abstract could be shorter than English.
%% 3. Please make sure that your input, including References, is on one page

%% Enabling Cyrillic input with koi8-r encoding:
%\usepackage[T2A]{fontenc}
%\usepackage[koi8-r]{inputenc}
%\usepackage[russian,english]{babel}
%\usepackage[english]{babel}

%% Standard packages and definitions:
\usepackage{amssymb}
\usepackage{latexsym}
\usepackage{amsmath}
\def\R{\mathbb{R}}
\def\C{\mathbb{C}}
\def\Z{\mathbb{Z}}
\def\Q{\mathbb{Q}}
\def\p{\partial}
\pagestyle{empty}

\begin{document}
\begin{center}

%% Title in English:
{\Large Sharp two-term Sobolev inequality and applications to the Lieb--Thirring estimates}

\bigskip

%% Author 1:
{\sc Alexei Ilyin}

%% Address:
{\small\it Keldysh Institute of Applied Mathematics, Moscow 125047, Russia}

%% Email:
{\small\rm ilyin@keldysh.ru}

\end{center}

\bigskip

%% Abstract in English:

For a function $\varphi\in H^1(\mathbb{R})$ the following inequality
is  well known
$$
\|\varphi\|_\infty^2\le \|\varphi\|\|\varphi'\|,
$$
where the norms on the right-hand side are the $L_2$-norms.
The constant $1$ in the  inequality is  sharp
and the unique extremal unction is
$
\varphi^*(x)=e^{-|x|}
$.
The same inequality clearly holds on a finite interval,
that is, for $\varphi\in H^1_0(0,L)$. However, since
$\varphi^*(x)>0$, no extremal functions exist.
The following result provides a sharp correction term
for this inequality (the correction term in the periodic case
was found in \cite{Zelik}).

\medskip
\noindent
{\bf Theorem 1.}
{\it Let $f\in H^1_0(0,L)$. Then
$
\ \|\varphi\|^2_\infty\le\|\varphi\|\|\varphi'\|\bigl(1-2e^{-\frac{L\|\varphi'\|}{\|\varphi\|}}\bigr).
$

\noindent
The  coefficients of the two terms on the right-hand side
are sharp and no extremal functions exist.}
\medskip



As in the periodic case considered in~\cite{JST}, this theorem makes
it possible to obtain a simultaneous bound for the negative
trace and the number of negative eigenvalues for the
1D Schr\"odinger eigenvalue problem on $(0,L)$
\begin{equation}
\label{Sch}
-y''_j-Vy_j=\nu_jy_j
\end{equation}
with Dirichlet boundary conditions $y(0)=y(L)=0$
and potential $V(x)\ge0$.

\medskip
\noindent
{\bf Theorem 2.}
{\it Suppose that there exist $N$ negative eigenvalues $\nu_j\le0$, %$\nu_j\ge0$,
$j=1,\dots,N$ of the operator~\eqref{Sch}. Then
both the negative trace and the number $N$ of negative
eigenvalues satisfy for any $\varepsilon\ge0$
$$
\sum_{j=1}^N|\nu_j|+N\cdot\frac{\pi^2}{L^2}
\biggl(\frac{c(\varepsilon)}{1+\varepsilon}\biggr)^{2}\le
\frac{2}{3\sqrt{3}}\cdot
(1+\varepsilon)\int_0^{L}V(x)^{3/2}dx,
$$
where
$
c(\varepsilon)=\min_{x\ge1}\bigl(\varepsilon x+2xe^{-\pi x}\bigr)
$.
}

\medskip
\noindent
{\it Remark.} We have  $c(\varepsilon)\ge\varepsilon$, and the optimal
$\varepsilon$ for the
bound involving only the number of negative eigenvalues $N$
is $\varepsilon=2$.




\begin{thebibliography}{7}



\bibitem{Zelik}M.V.\,Bartuccelli, J.\,Deane, and
S.V.\,Zelik, \emph{Asymptotic expansions and extremals for
the critical Sobolev and Gagliardo-Nirenberg inequalities on a
torus.} arXiv:1012.2061 (2010).

\bibitem{JST} A.A.Ilyin,
\emph{Lieb--Thirring inequalities on some manifolds}, Journal
of Spectral Theory
\textbf{2}:1 (2012), 1--22.

\end{thebibliography}

\end{document}
