\documentclass[10pt,a4paper]{article}

%% Title/abstract template for the conference in honour of Mark Vishik
%% The original is at http://www.dynamics.iitp.ru/vishik/abstract.tex
%%
%% Instructions:
%% 1. Title and abstract should be in English
%% 2. You are welcome to include the Russian translation (optional).
%%    Russian abstract could be shorter than English.
%% 3. Please make sure that your input, including References, is on one page

%% Enabling Cyrillic input with koi8-r encoding:
%\usepackage[T2A]{fontenc}
%\usepackage[koi8-r]{inputenc}
%\usepackage[russian,english]{babel}
%\usepackage[english]{babel}

%% Standard packages and definitions:
\usepackage{amssymb}
\usepackage{latexsym}
\usepackage{amsmath}
\def\R{\mathbb{R}}
\def\C{\mathbb{C}}
\def\Z{\mathbb{Z}}
\def\Q{\mathbb{Q}}
\def\p{\partial}
\pagestyle{empty}

\begin{document}
\begin{center}

%% Title in English:
{\Large Global well-posedness of an inviscid three-dimensional pseudo-Hasegawa-Mima model}

\bigskip

%% Author 1:
{\sc Chongsheng Cao}

%% Address:
{\small\it Florida International University, Miami, FL 33199, USA}

%% Email:
{\small\rm caoc@fiu.edu}


\bigskip

%% Author 2:
{\sc Aseel Farhat}

{\small\it UC -- Irvine, CA 92697, USA}

{\small\rm  afarhat@math.uci.edu}

\bigskip

%% Author 3:
{\sc Edriss S. Titi}

{\small\it UC -- Irvine, CA 92697, USA; Weizmann Institute of Science, Rehovot 76100, Israel}

{\small\rm  etiti@math.uci.edu}

\end{center}

\bigskip

%% Abstract in English:
The three-dimensional inviscid Hasegawa-Mima model is one of the fundamental models that describe plasma turbulence. The model also appears as a simplified reduced Rayleigh-B\'enard convection model. The mathematical analysis of the Hasegawa-Mima equation is challenging  due to the absence of any smoothing viscous terms, as well as to the presence of an analogue of the vortex stretching terms. In this talk, we introduce and study a model which is inspired by the inviscid Hasegawa-Mima model, which we call a pseudo-Hasegawa-Mima model. The introduced model is easier to investigate analytically than the original inviscid Hasegawa-Mima model, as it has a nicer mathematical structure. The resemblance between this model and the Euler equations of inviscid incompressible fluids inspired us to adapt the techniques and ideas introduced for the two-dimensional and the three-dimensional Euler equations to prove the global existence and uniqueness of solutions for our model. This is in addition to proving and implementing a new technical logarithmic inequality, generalizing the Brezis-Gallouet and the Berzis-Wainger inequalities. Moreover, we prove the continuous dependence on initial data of solutions for the pseudo-Hasegawa-Mima model. These are the first results on existence and uniqueness of solutions for a model that is related to the three-dimensional inviscid Hasegawa-Mima equations.




\end{document}
