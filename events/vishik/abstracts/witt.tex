\documentclass[10pt,a4paper]{article}

%% Title/abstract template for the conference in honour of Mark Vishik
%% The original is at http://www.dynamics.iitp.ru/vishik/abstract.tex
%%
%% Instructions:
%% 1. Title and abstract should be in English
%% 2. You are welcome to include the Russian translation (optional).
%%    Russian abstract could be shorter than English.
%% 3. Please make sure that your input, including References, is on one page

%% Enabling Cyrillic input with koi8-r encoding:
%\usepackage[T2A]{fontenc}
%\usepackage[koi8-r]{inputenc}
%\usepackage[russian,english]{babel}
%\usepackage[english]{babel}
\usepackage[T1]{fontenc}
\usepackage[latin1]{inputenc}
\usepackage[english]{babel}


%% Standard packages and definitions:
\usepackage{amssymb}
\usepackage{latexsym}
\usepackage{amsmath}
\def\R{\mathbb{R}}
\def\C{\mathbb{C}}
\def\Z{\mathbb{Z}}
\def\Q{\mathbb{Q}}
\def\p{\partial}
\pagestyle{empty}

\begin{document}
\begin{center}

%% Title in English:
{\Large On a class of degenerate pseudodifferential operators and applications to mixed-type PDEs}

\bigskip

%% Author 1:
{\sc Ingo Witt}

%% Address:
{\small\it Universit{\"a}t G{\"o}ttingen, G{\"o}ttingen, Germany}

%% Email:
{\small\rm iwitt@uni-math.gwdg.de}

\end{center}

\bigskip

%% Abstract in English:
In a series of papers in 1969/70, Vishik and Grushin \cite{VG69, VG70}
introduced a class of degenerate pseudodifferential operators roughly
being of the form
\[
|x|^{b-(l_\ast+1)m} P(x,y,|x|D_x,|x|^{l_\ast+1}D_y),
\]
where $m$ is the order of the operator and $b\in{\mathbb R}$ is a
parameter ($b=m l_\ast$ in the work of Vishik and Grushin). Here,
$l_\ast\in {\mathbb Q}$, $l_\ast>0$, $x\in{\mathbb R}^q$, $y\in
{\mathbb R}^d$, and $P(x,y,\xi,\eta)$ is a pseudodifferential
symbol. A basic example is the Tricomi operator $\partial_x^2 + x
\Delta_y$, where $l_\ast=1/2$, $q=1$, $m=2$, and $b=1$.

In this talk, we develop a pseudodifferential calculus for such
operators including a full symbol calculus, discuss elliptic boundary
problems \cite{W03} and the hyperbolic Cauchy problem \cite{DW02} in
this class, and eventually present applications to 2D mixed
elliptic-hyperbolic equations (where $q=d=1$) with variable
coefficients. In the latter case, the problem is reduced to a
cone-degenerate elliptic problem in the interface (which is an
interval), where the equations under study change type.

\begin{thebibliography}{9}

\bibitem{DW02} M.~Dreher and I.~Witt, \emph{Edge {S}obolev spaces and
weakly hyperbolic equations}, Ann. Mat. Pura Appl. (4),
\textbf{180} (2002), 451--482.

\bibitem{VG69} M.I.~Vishik and V.V.~Grushin, \emph{A certain class of
degenerate elliptic equations of higher orders}, Mat. Sb. (N.S.)
\textbf{79}~(\textbf{121}) (1969), 3--36 (in Russian).

\bibitem{VG70} M.I.~Vishik and V.V.~Grushin, \emph{Degenerate elliptic
differential and pseudo\-differential operators}, Uspehi Mat. Nauk
\textbf{25} (1970), 29--56 (in Russian).

\bibitem{W03} I.~Witt, \emph{A calculus for a class of finitely
degenerate pseudodifferential operators}. In: Evolution equations,
volume 60 of Banach Center Publ., Polish Acad. Sci., 2003,
pp.~161-189.

\end{thebibliography}

\end{document}
