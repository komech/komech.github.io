%% LyX 2.0.2 created this file.  For more info, see http://www.lyx.org/.
%% Do not edit unless you really know what you are doing.
\documentclass[10pt,a4paper,english]{article}
\usepackage[T2A]{fontenc}
\pagestyle{empty}
\usepackage{amsmath}
\usepackage{amssymb}

\makeatletter

%%%%%%%%%%%%%%%%%%%%%%%%%%%%%% LyX specific LaTeX commands.
\special{papersize=\the\paperwidth,\the\paperheight}


%%%%%%%%%%%%%%%%%%%%%%%%%%%%%% User specified LaTeX commands.


%% Title/abstract template for the conference in honour of Mark Vishik
%% The original is at http://www.dynamics.iitp.ru/vishik/abstract.tex
%%
%% Instructions:
%% 1. Title and abstract should be in English
%% 2. You are welcome to include the Russian translation (optional).
%%    Russian abstract could be shorter than English.
%% 3. Please make sure that your input, including References, is on one page

%% Enabling Cyrillic input with koi8-r encoding:
%\usepackage[koi8-r]{inputenc}
\usepackage[english]{babel}


%% Standard packages and definitions:
\usepackage{latexsym}\def\R{\mathbb{R}}
\def\C{\mathbb{C}}
\def\Z{\mathbb{Z}}
\def\Q{\mathbb{Q}}

\makeatother

\usepackage{babel}
\begin{document}
\begin{center}
%% Title in English:
{\Large New phenomena in large systems of ODE and classical models of DC} 

\bigskip

%% Author 1:
{\sc Vadim Malyshev}

%% Address:
{\small\it Moscow State University, Moscow 119992, Russia}

%% Email:
{\small\rm malyshev2@yahoo.com}

\end{center}

\bigskip


%% Abstract in English:


\vspace*{3mm}
We consider the system
\[
M\frac{d^{2}x_{i}}{dt^{2}}=-\frac{\partial U}{\partial x_{i}}+F(x_{i})-A\frac{dx_{i}}{dt},i=1,...,N
\]
of $N$ ordinary differential equations describing Newtonian dynamics
of $N$ particles (electrons), initially at the points 
\[
x_{1}(0)<x_{2}(0)<...<x_{N}(0),
\]
on the interval $[0,L)\in R$ with periodic boundary conditions, that
is on the circle of length $L$. Here $M>0,A\geq0$ are the parameters,
$F(x)$ is the external force, and 
\[
U(x_{1},...,x_{N})=\sum_{i=1}^{N}\frac{\alpha}{|x_{i+1}-x_{1}|}.\alpha>0,
\]
(where of course $x_{N+1}\equiv x_{1}$) is the Coulomb repulsive
interaction between nearest neighbors.

We review new results concerning this system: fixed points, quasi-homogeneous
regime (Ohm's law) and very fast propagation of the ``effective''
external field, which is initially zero on the most part of the circle.

All these phenomena are closely related to many problems with DC (direct
electric current), that the statistical physics was unable to understand.
The following is a picturesque description of one of DC enigmas in
the famous Feynman lectures, v. 6, pp. 33-34: ``The force pushes
the electrons along the wire. But why does this move the galvanometer,
whis is so far from the force? Because when the electrons which feel
the magnetic force try to move, they push - by electric repulsion
- the electrons a little farther down the wire; they, in turn, repel
the electrons a little farther on, and so on for a long distance.
An amazing thing. It was so amazing to Gauss and Weber - who first
built a galvanometer - that they tried to see how far the forces in
the wire would go. They strung the wire all the way across the city.''
\end{document}
