\documentclass[10pt,a4paper]{article}

%% Title/abstract template for the conference in honour of Mark Vishik
%% The original is at http://www.dynamics.iitp.ru/vishik/abstract.tex
%%
%% Instructions:
%% 1. Title and abstract should be in English
%% 2. You are welcome to include the Russian translation (optional).
%%    Russian abstract could be shorter than English.
%% 3. Please make sure that your input, including References, is on one page

%% Enabling Cyrillic input with koi8-r encoding:
%\usepackage[T2A]{fontenc}
%\usepackage[koi8-r]{inputenc}
%\usepackage[russian,english]{babel}
%\usepackage[english]{babel}

%% Standard packages and definitions:
\usepackage{amssymb}
\usepackage{latexsym}
\usepackage{amsmath}
\def\R{\mathbb{R}}
\def\C{\mathbb{C}}
\def\Z{\mathbb{Z}}
\def\Q{\mathbb{Q}}
\def\p{\partial}
\pagestyle{empty}

\begin{document}
\begin{center}

%% Title in English:
{\Large Vishik's approach to general boundary value problems for elliptic operators. Recent development}

\bigskip

%% Author 1:
{\sc Mark Malamud}

%% Address:
{\small\it Institute of Applied Mathematics and Mechanics, Donetsk 83114, Ukraine}

%% Email:
{\small\rm mmm@telenet.dn.ua}

\end{center}

\bigskip

%% Abstract in English:


%%Its influence and further development.

In his pioneering paper \cite{Vis52} M.I. Vishik proposed a new
approach to the extension theory of symmetric operators as well as
dual pairs of operators in a Hilbert space. In the framework of
this approach the proper extensions are parameterized in terms of
(abstract) boundary conditions. Moreover, he applied
general constructions to investigate  %%some spectral properties
the properties of solvability and complete solvability  of
boundary value problems for (not necessarily symmetric) elliptic
operators on bounded domains.


During three last decades this approach has been formalized in the
concept of boundary triplets for dual pairs of operators and 
elaborated in great detail.  The revival of interest to this
approach  has been motivated by numerous applications to boundary
value problems for differential and difference operators (see for
instance publications \cite{Grubb68}, \cite{BroGruWoo09},
\cite{KosMal10}, \cite{Mal10} and references therin).


I plan to recall the main results and basic constructions of the
Vishik's paper  \cite{Vis52} as well as  to discuss  its influence
on development of the extensions theory.
%%of symmetric operators as well as dual pairs of operators will be discussed.

Next I plan to discuss  applications of to elliptic boundary value
problems in domain with compact boundary. Some spectral properties
of different realizations of elliptic differential expressions
will be discussed too.


\begin{thebibliography}{7}

\bibitem{Vis52}
M.I. Vishik. \emph{On general boundary problems for elliptic
differential equations}. {Am. Math. Soc., Transl., II. Ser.}.
\textbf{24}(1952), 107--172.

\bibitem{Grubb68}
G.~Grubb. \emph{A characterization of the non-local boundary value
problems associated with an elliptic operator}.  {Ann. Scuola
Norm. Sup. Pisa}. \textbf{3, 22}(1968), 425--513.


\bibitem{BroGruWoo09}
B. M. Brown, G.Grubb, I. G.Wood, \emph{M-functions for closed
extensions of adjoint pairs of operators with applications to
elliptic boundary problems}.  {Math. Nachr.} \textbf{282, No.3}
(2009), 314--347.

\bibitem{KosMal10}
A.~S. Kostenko and M.~M. Malamud. \emph{1-D Schr\"odinger
operators with local point interactions on a discrete set}. {J.
Differ. Equations}. \textbf{249(2)}(2010), 253--304.


\bibitem{Mal10}
M.~M. Malamud.  \emph{Spectral theory of elliptic operators in
exterior domains}. {Russ. J. Math. Phys.}, \textbf{17(1)}(2010),
96--125.

\end{thebibliography}

\end{document}
