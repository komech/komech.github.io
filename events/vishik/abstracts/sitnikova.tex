\documentclass[10pt,a4paper]{article}

%% Title/abstract template for the conference in honour of Mark Vishik
%% The original is at http://www.dynamics.iitp.ru/vishik/abstract.tex
%%
%% Instructions:
%% 1. Title and abstract should be in English
%% 2. You are welcome to include the Russian translation (optional).
%%    Russian abstract could be shorter than English.
%% 3. Please make sure that your input, including References, is on one page

%% Enabling Cyrillic input with koi8-r encoding:
\usepackage[T2A]{fontenc}
\usepackage[koi8-r]{inputenc}
\usepackage[russian,english]{babel}

%% Standard packages and definitions:
\usepackage{amssymb}
\usepackage{latexsym}
\usepackage{amsmath}
\def\R{\mathbb{R}}
\def\C{\mathbb{C}}
\def\Z{\mathbb{Z}}
\def\Q{\mathbb{Q}}
\def\p{\partial}
\pagestyle{empty}

\begin{document}
\begin{center}

%% Title in English:
{\Large Eigenfunction of the Laplace operator in a tetrahedron}

\bigskip

%% Author 1:
{\sc Elena Sitnikova}

%% Address:
{\small\it Moscow State University of Civil Engineering, Moscow 129337, Russia}

%% Email:
{\small\rm 301064@mail.ru}


\end{center}

\bigskip

%% Abstract in English:

Let T be an open and regular triangular pyramid (tetrahedron) in the space $\R^3$ with a
boundary $\p\Gamma$.
Let $\alpha$, $\beta$, $\gamma$, $\sigma$
are barycentric coordinates of a point $(x, y, z) \in \R^3$
with respect to
tetrahedron $T$ which can be expressed in the variables $x,\, y,\, z$.

\bigskip

\noindent
{\bf Theorem.}
The function $w =\sin(\alpha\pi/2)\sin(\beta\pi/2)\sin(\gamma\pi/2)\sin(\sigma\pi/2)$
is the eigenfunction of the
Laplace operator $\Delta\equiv
\frac{\p^2}{\p x^2}+\frac{\p^2}{\p y^2}+\frac{\p^2}{\p z^2}$
in $T$.
The function $w$ satisfies conditions: $w >0$ in $T$
and $w=0$ on $\p T$.

Let $\Pi$ be unlimited cylinder in the space $\R^4$
which a cross-section with hyperplane is a
quadrangular pyramid with edges of unit length (one-half of the octahedron). Let $L$ be a second
order linear differential operator in divergence form which uniformly elliptic with bounded
measurable coefficients and $\eta$ is its ellipticity constant. Let $u$ be a solution of he mixed boundary
value problem in $\Pi$ for the equation $Lu=0$ ($u>0$) with homogeneous Dirichlet and Neumann data
on the boundary of the cylinder. Our theorem allows us to continue this solution from the
cylinder $\Pi$ to the whole space $\R^4$ with the same ellipticity constant $\eta$.

This continuation allows us to prove a number of theorems about growth of the solution u in
the cylinder $\Pi$.

The idea of using barycentric coordinates is taken from paper of A.P.\,Brodnikov, where it is
used for the finding of eigenfunction of the Laplace operator in the triangle.
Eigenfunction of the Laplace operator in hypertetrahedron from $\R^4$ and in $n+1$-dimensional
simplex from $\R^n$ ($n\ge 2$) were constructed by the author.


\end{document}
