\documentclass[10pt,a4paper]{article}

%% Title/abstract template for the conference in honour of Mark Vishik
%% The original is at http://www.dynamics.iitp.ru/vishik/abstract.tex
%%
%% Instructions:
%% 1. Title and abstract should be in English
%% 2. You are welcome to include the Russian translation (optional).
%%    Russian abstract could be shorter than English.
%% 3. Please make sure that your input, including References, is on one page

%% Enabling Cyrillic input with koi8-r encoding:
%\usepackage[T2A]{fontenc}
%\usepackage[koi8-r]{inputenc}
%\usepackage[russian,english]{babel}
%\usepackage[english]{babel}

%% Standard packages and definitions:
\usepackage{amssymb}
\usepackage{latexsym}
\usepackage{amsmath}
\def\R{\mathbb{R}}
\def\C{\mathbb{C}}
\def\Z{\mathbb{Z}}
\def\Q{\mathbb{Q}}
\def\p{\partial}
\newtheorem{theorem}{Theorem}

\pagestyle{empty}

\begin{document}
\begin{center}

%% Title in English:
{\Large On the 2-point problem for the Lagrange-Euler equation}

\bigskip

%% Author 1:
{\sc Alexander Shnirelman}

%% Address:
{\small\it Concordia University, Montreal, Canada}

%% Email:
{\small\rm shnirel@mathstat.concordia.ca}

\end{center}

\bigskip

Consider the motion of ideal incompressible fluid in a bounded
domain (or on a compact Riemannian manifold). The configuration space of
the fluid is the group of volume preserving diffeomorphisms of the flow
domain, and the flows are geodesics on this infinite-dimensional group
where the metric is defined by the kinetic energy. The geodesic equation is
the Lagrange-Euler equation. The problem usually studied is the initial
value problem, where we look for a geodesic with given initial fluid
configuration and initial velocity field. In this talk we consider a
different problem: find a geodesic connecting two given fluid
configurations. The main result is the following

\smallskip

\noindent{\bf Theorem:}
{\it Suppose the flow domain is a 2-dimensional torus. Then for any two
fluid configurations there exists a geodesic connecting them. This means
that, given arbitrary fluid configuration (diffeomorphism), we can "push"
the fluid along some initial velocity field, so that by time one the fluid,
moving according to the Lagrange-Euler equation, assumes the given
configuration.}

\smallskip

This theorem looks superficially like the Hopf-Rinow theorem for
finite-dimensional Riemannian manifolds. In fact, these two theorems have
almost nothing in common. In our case, unlike the Hopf-Rinow theorem, the
geodesic is not, in general case, the shortest curve connecting the
endpoints (fluid configurations). Moreover, the length minimizing curve can
not exist at all, while the geodesic always exists.

The proof is based on some ideas of global analysis (Fredholm quasilinear
maps) and microlocal analysis of the Lagrange-Euler equation (which may be
called a ``microglobal analysis'').

\end{document}
