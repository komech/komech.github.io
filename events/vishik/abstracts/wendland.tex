\documentclass[10pt,a4paper]{article}

%% Title/abstract template for the conference in honour of Mark Vishik
%% The original is at http://www.dynamics.iitp.ru/vishik/abstract.tex
%%
%% Instructions:
%% 1. Title and abstract should be in English
%% 2. You are welcome to include the Russian translation (optional).
%%    Russian abstract could be shorter than English.
%% 3. Please make sure that your input, including References, is on one page

%% Enabling Cyrillic input with koi8-r encoding:
\usepackage[T2A]{fontenc}
\usepackage[koi8-r]{inputenc}
\usepackage[russian,english]{babel}

%% Standard packages and definitions:
\usepackage{amssymb}
\usepackage{latexsym}
\usepackage{amsmath}
\def\R{\mathbb{R}}
\def\C{\mathbb{C}}
\def\Z{\mathbb{Z}}
\def\Q{\mathbb{Q}}
\pagestyle{empty}

\begin{document}
\begin{center}

%% Title in English:
{\Large On the Gauss problem with Riesz  potential}

\bigskip

%% Author 1:
{\sc Wolfgang L. Wendland}

%% Address:
{\small\it Universit{\"a}t Stuttgart, Germany}

%% Email:
{\small\rm wendland@mathematik.uni-stuttgart.de}


\end{center}

\bigskip

%% Abstract in English:

This is a lecture on joint work with H.~Harbrecht (U. Basel,
Switzerland), G.~Of (TU. Graz, Austria) and
N.~Zorii (Nat. Academy Sci. Kiev, Ukraine).

In $\mathbb R^n$, $n\geqslant 2$, we study the constructive and
numerical solution of minimizing the energy relative to the Riesz
kernel $|{\bf x}-{\bf y}|^{\alpha-n}$, where $1<\alpha<n$, for the
Gauss variational problem, considered for finitely many compact,
mutually disjoint, boundaryless $(n-1)$--dimensional
Lipschitz manifolds $\Gamma_\ell$, $\ell\in L$, each $\Gamma_\ell$
being charged with Borel measures with the sign $\alpha_\ell=\pm1$
prescribed. We show that the Gauss variational problem over an
affine cone of Borel measures can alternatively be formulated as a
minimum problem over an affine cone of surface distributions
belonging to the Sobolev--Slobodetski space
$H^{-{\varepsilon}/{2}}(\Gamma)$, where $\varepsilon:=\alpha-1$ and
$\Gamma:=\bigcup_{\ell\in L}\,\Gamma_\ell$. This allows the
application of simple layer boundary integral operators on~$\Gamma$
and, hence, a penalty approximation. A corresponding numerical
method is based on the Galerkin--Bubnov discretization with
piecewise constant boundary elements. For $n=3$ and  $\alpha=2$, multipole
approximation and in the case  $1<\alpha<3=n$  wavelet matrix compression is
applied to sparsify the system matrix.  Numerical results are
presented to illustrate the approach.


\begin{thebibliography}{6}

\bibitem{=WZ}
G.~Of, W.L.~Wendland and N.~Zorii:
\textsl{On the numerical solution of minimal energy problems.}
Complex Variables and Elliptic Equations \textbf{55} (2010)
991--1012.

\bibitem{HWZ}
H.~Harbrecht, W.L.~~Wendland and N.~Zorii:
\textsl{On Riesz minimal energy problems.}
Preprint Series Stuttgart Research Centre for Simulation Technology
(SRC Sim Tech) Issue No. 2010--80.

\end{thebibliography}

\end{document}
