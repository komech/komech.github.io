%% Title/abstract template for the conference in honour of Mark Vishik
%% The original is at http://www.dynamics.iitp.ru/vishik/abstract.tex
%% Instructions:
%% 1. Title and abstract should be in English
%% 2. You are welcome to include the Russian translation (optional).
%%    Russian abstract could be shorter than English.
%% 3. Please make sure that your input, including References, is on one page
%% Enabling Cyrillic input with koi8-r encoding:
%% Standard packages and definitions:
%%\usepackage[T2A]{fontenc}
%%\usepackage[koi8-r]{inputenc}
%%\usepackage[russian,english]{babel}


\documentclass[10pt,a4paper]{article}
%%%%%%%%%%%%%%%%%%%%%%%%%%%%%%%%%%%%%%%%%%%%%%%%%%%%%%%%%%%%%%%%%%%%%%%%%%%%%%%%%%%%%%%%%%%%%%%%%%%%%%%%%%%%%%%%%%%%%%%%%%%%%%%%%%%%%%%%%%%%%%%%%%%%%%%%%%%%%%%%%%%%%%%%%%%%%%%%%%%%%%%%%%%%%%%%%%%%%%%%%%%%%%%%%%%%%%%%%%%%%%%%%%%%%%%%%%%%%%%%%%%%%%%%%%%%
\usepackage{amssymb}
\usepackage{latexsym}
\usepackage{amsmath}

\setcounter{MaxMatrixCols}{10}
%TCIDATA{OutputFilter=LATEX.DLL}
%TCIDATA{Version=5.50.0.2953}
%TCIDATA{<META NAME="SaveForMode" CONTENT="1">}
%TCIDATA{BibliographyScheme=Manual}
%TCIDATA{LastRevised=Tuesday, February 28, 2012 13:06:23}
%TCIDATA{<META NAME="GraphicsSave" CONTENT="32">}

\def\R{\mathbb{R}}
\def\C{\mathbb{C}}
\def\Z{\mathbb{Z}}
\def\Q{\mathbb{Q}}
\pagestyle{empty}
%%\input{tcilatex}
\begin{document}


\begin{center}
%% Title in English:
{\large Relativistic point dynamics and Einstein's formula as a property of localized solutions of a nonlinear Klein-Gordon equation}

\smallskip

%% Author 1:
{\sc Anatoli Babin}

%% Address:
{\small\it UC -- Irvine, Irvine CA 92617, USA}

%% Email:
{\small\rm ababine@uci.edu}

\smallskip

%% Author 2:
{\sc Alexander Figotin}

{\small\it UC -- Irvine, Irvine CA 92617, USA}

{\small\rm afigotin@uci.edu}
\end{center}

%\medskip

%% Abstract in English:
Relativistic mechanics includes the relativistic dynamics of a mass point
and the relativistic field theory. In a relativistic field theory the
relativistic field dynamics is derived from a relativistic covariant
Lagrangian, such a theory allows to define the total energy and momentum,
forces and their densities but does not provide a canonical way to define
the mass, position or velocity for the system. For a closed system \emph{%
without external forces} the total momentum has a simple form $\mathbf{P}=M%
\mathbf{v}$ where $\mathbf{v}$ is a \emph{constant velocity}, allowing to
define naturally the total mass $M$ and to derive from the Lorentz
invariance Einstein's energy-mass relation $E=Mc^{2}$ with $M=m_{0}\gamma ,$
with $\gamma $ the Lorentz factor and $m_{0}$ the rest mass;
according to Einstein's formula the rest mass is determined by the internal
energy of the system.

The relativistic dynamics of a mass point is described by a relativistic
version of Newton's equation where the rest mass $m_{0}$ of a point is
prescribed; in Newtonian mechanics the mass $M$ reveals itself in \emph{%
accelerated motion} as a measure of inertia which relates the point
acceleration to the external force. The question which we address is the
following: Is it possible to construct a mathematical model where the
internal energy of a system affects its acceleration in an external force
field as the inertial mass does in Newtonian mechanics? 

We construct a model which allows to consider in the same framework the
uniform motion in the absence of external forces (a closed system) and the
accelerated motion caused by external fields; the internal energy is present
both in uniform and accelerating regimes. The model is based on the
nonlinear Klein-Gordon (KG)\ equation which is a part of our theory of
distributed charges interacting with electromagnetic (EM) fields, \cite{BF5}-%
\cite{BF8}. We prove that if a sequence of solutions of a KG\ equation
concentrates at a trajectory $\mathbf{r}(t)$ and their local energies
converge to $E(t)$ then the trajectory satisfies the relativistic version of
Newton's equation where the mass is determined in terms of the energy by
Einstein's formula, and the EM forces are determined by the coefficients of
the KG equation. We prove that the concentration assumptions hold for the
case of a general rectilinear accelerated motion.
\ \vskip -0.6cm\ 

\begin{thebibliography}{9}
\bibitem{BF5} Babin A. and Figotin A., J. Stat. Phys., \textbf{138}:
912--954, (2010).

\bibitem{BF6} Babin A. and Figotin A., DCDS A, \textbf{27}(4), 1283-1326,
(2010).

\bibitem{BF7} Babin A. and Figotin A., Found. Phys., \textbf{41}: 242--260,
(2011).

\bibitem{BF8} Babin A. and Figotin A., arXiv:1110.4949v3; Found. Phys, 2012.
\end{thebibliography}

\end{document}
