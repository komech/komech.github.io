\documentclass[10pt,a4paper]{article}

%% Title/abstract template for the conference in honour of Mark Vishik
%% The original is at http://www.dynamics.iitp.ru/vishik/abstract.tex
%%
%% Instructions:
%% 1. Title and abstract should be in English
%% 2. You are welcome to include the Russian translation (optional).
%%    Russian abstract could be shorter than English.
%% 3. Please make sure that your input, including References, is on one page

%% Enabling Cyrillic input with koi8-r encoding:
%\usepackage[T2A]{fontenc}
%\usepackage[koi8-r]{inputenc}
%\usepackage[russian,english]{babel}
%\usepackage[english]{babel}

%% Standard packages and definitions:
\usepackage{amssymb}
\usepackage{latexsym}
\usepackage{amsmath}
\def\R{\mathbb{R}}
\def\C{\mathbb{C}}
\def\Z{\mathbb{Z}}
\def\Q{\mathbb{Q}}
\def\p{\partial}
\pagestyle{empty}

\begin{document}
\begin{center}

%% Title in English:
{\Large Trajectory attractors for equations of mathematical physics}

\bigskip

%% Author 1:
{\sc Vladimir Chepyzhov}

%% Address:
{\small\it Institute for Information Transmission Problems, Moscow 101447, Russia}

%% Email:
{\small\rm chep@iitp.ru}

\end{center}

\bigskip

%% Abstract in English:

The report is based on joint works with M.I.Vishik.

We describe the method of trajectory dynamical systems and
trajectory attractors and we apply this approach to the study of
the limiting asymptotic behaviour of solutions of non-linear
evolution equations. This method is especially useful in the study
of dissipative equations of mathematical physics for which the
corresponding Cauchy initial-value problem has a global (weak)
solution with respect to the time but the uniqueness of this
solution either has not been established or does not hold. An
important example of such an equation is the 3D Navier--Stokes
system in a bounded domain  (see \cite{CVbook}). In such a
situation one cannot use directly the classical scheme of
construction of a dynamical system in the phase space of initial
conditions of the Cauchy problem of a given equation and find a
global attractor of this dynamical system. Nevertheless, for such
equations it is possible to construct a trajectory dynamical system
and investigate a trajectory attractor of the corresponding
translation semigroup.

This universal method is applied for various types of equations
arising in mathematical physics: for general dissipative
reaction-diffusion systems, for the 3D Navier--Stokes system, for
dissipative wave equations, for non-linear elliptic equations in
cylindrical domains, and for other equations and systems. Special
attention is given to using the method of trajectory attractors in
approximation and perturbation problems arising in complicated
models of mathematical physics.

The work partially supported by the Russian Foundation of Basic
Researches (Projects no.\ 11-01-00339 and 10-01-00293).


\begin{thebibliography}{99}


\bibitem{CVbook}V.V. Chepyzhov, M.I. Vishik, \emph{Attractors for
Equations of Mathematical Physics} Amer. Math. Soc. Colloq.
Publ., \textbf{49}, Amer.Math. Soc., Providence, RI, 2002.

\end{thebibliography}

\end{document}
