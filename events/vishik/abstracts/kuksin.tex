\documentclass[10pt,a4paper]{article}

%% Title/abstract template for the conference in honour of Mark Vishik
%% The original is at http://www.dynamics.iitp.ru/vishik/abstract.tex
%%
%% Instructions:
%% 1. Title and abstract should be in English
%% 2. You are welcome to include the Russian translation (optional).
%%    Russian abstract could be shorter than English.
%% 3. Please make sure that your input, including References, is on one page

%% Enabling Cyrillic input with koi8-r encoding:
%\usepackage[T2A]{fontenc}
%\usepackage[koi8-r]{inputenc}
%\usepackage[russian,english]{babel}
%\usepackage[english]{babel}

%% Standard packages and definitions:
\usepackage{amssymb}
\usepackage{latexsym}
\usepackage{amsmath}
\def\R{\mathbb{R}}
\def\C{\mathbb{C}}
\def\Z{\mathbb{Z}}
\def\Q{\mathbb{Q}}
\def\p{\partial}
\pagestyle{empty}

\begin{document}
\begin{center}

%% Title in English:
{\Large Around the Cauchy-Kowalevski theorem}

\bigskip

%% Author 1:
{\sc Sergei Kuksin}

%% Address:
{\small\it Ecole Polytechnique, Paris}

%% Email:
{\small\rm kuksin@gmail.com}

\end{center}

\bigskip

I will present a general approach which allows to prove the propagation of
analyticity for solutions of various classes of quasilinear and nonlinear PDEs.
In particular, it implies that under  the assumptions of the Cauchy-Kowalevski
or Ovsiannikov-Nirenberg theorems classical solutions stay analytic till they
+exist. This is a joint work with N. Nadirashvili.

\end{document}
