%% Title/abstract template for the conference in honour of Mark Vishik
%% The original is at http://www.dynamics.iitp.ru/vishik/abstract.tex
%% Instructions:
%% 1. Title and abstract should be in English
%% 2. You are welcome to include the Russian translation (optional).
%%    Russian abstract could be shorter than English.
%% 3. Please make sure that your input, including References, is on one page
%% Enabling Cyrillic input with koi8-r encoding:
%\usepackage[T2A]{fontenc}
%\usepackage[koi8-r]{inputenc}
%\usepackage[russian,english]{babel}
%\usepackage[english]{babel}
%% Standard packages and definitions:


\documentclass[10pt,a4paper]{article}
%%%%%%%%%%%%%%%%%%%%%%%%%%%%%%%%%%%%%%%%%%%%%%%%%%%%%%%%%%%%%%%%%%%%%%%%%%%%%%%%%%%%%%%%%%%%%%%%%%%%%%%%%%%%%%%%%%%%%%%%%%%%%%%%%%%%%%%%%%%%%%%%%%%%%%%%%%%%%%%%%%%%%%%%%%%%%%%%%%%%%%%%%%%%%%%%%%%%%%%%%%%%%%%%%%%%%%%%%%%%%%%%%%%%%%%%%%%%%%%%%%%%%%%%%%%%
\usepackage{amsfonts}
\usepackage{amssymb}
\usepackage{latexsym}
\usepackage{amsmath}

\setcounter{MaxMatrixCols}{10}
%TCIDATA{OutputFilter=Latex.dll}
%TCIDATA{Version=5.50.0.2953}
%TCIDATA{<META NAME="SaveForMode" CONTENT="1">}
%TCIDATA{BibliographyScheme=Manual}
%TCIDATA{LastRevised=Friday, March 09, 2012 15:28:01}
%TCIDATA{<META NAME="GraphicsSave" CONTENT="32">}
%TCIDATA{Language=American English}
%TCIDATA{CSTFile=LaTeX article (bright).cst}

\pagestyle{empty}

\begin{document}


\begin{center}
%% Title in English:
{\Large Negative eigenvalues of two-dimensional Schr\"{o}dinger operators}

\bigskip

%% Author 1:
{\sc Alexander Grigoryan}

%% Address:
{\small\it University of Bielefeld, 33501 Bielefeld, Germany}

%% Email:
{\small\rm grigor@math.uni-bielefeld.de}

\end{center}

\bigskip

%% Abstract in English:
Given a non-negative $L_{loc}^{1}$ function $V\left( x\right) $ on $\mathbb{R%
}^{n}$, consider the Schr\"{o}dinger operator
$
H_{V}=-\Delta -V
$
where $\Delta =\sum_{k=1}^{n}\frac{\partial ^{2}}{\partial x_{k}^{2}}$ is
the Laplace operator. More precisely, $H_{V}$ is defined as a form sum of $%
-\Delta $ and $-V$, so that, under certain assumptions about $V$, the
operator $H_{V}$ is self-adjoint in $L^{2}\left( \mathbb{R}^{n}\right) $.

Denote by $\mathrm{Neg}\left( V\right) $ the number of non-positive
eigenvalues of $H_{V}$ (counted with multiplicity), assuming that its
spectrum in $(-\infty ,0]$ is discrete. For example, the latter is the case
when $V\left( x\right) \rightarrow 0$ as $x\rightarrow \infty $. We are are
interested in obtaining estimates of $\mathrm{Neg}\left( V\right) $ in terms
of the potential $V$ in the case $n=2.$

For the operator $H_{V}$ in $\mathbb{R}^{n}$ with $n\geq 3$ a celebrated
inequality of Cwikel-Lieb-Rozenblum says that%
\begin{equation}
\mathrm{Neg}\left( V\right) \leq C_{n}\int_{\mathbb{R}^{n}}V\left( x\right)
^{n/2}dx.  \label{CLR}
\end{equation}%
For $n=2$ this inequality is not valid. Moreover, no weighted $L^{1}$-norm
of $V$ can provide an upper bound for $\mathrm{Neg}\left( V\right) .$ In fact,
in the case $n=2$ instead of the upper bounds, the lower bound in (\ref{CLR}%
) is true.

The main result is the estimate (\ref{NegIn}) below that was obtained
jointly with N.Nadira\-shvili. For any $n\in \mathbb{Z}$, set 
\[
U_{n} =
\left\{
\begin{array}{l}
\{e^{2^{n-1}}<\left\vert x\right\vert <e^{2^{n}}\},\ \ n>0, \\
\{e^{-1}<\left\vert x\right\vert <e\}, \ \ n=0,
\\
\{e^{-2^{\left\vert n\right\vert }}<\left\vert x\right\vert
<e^{-2^{\left\vert n\right\vert -1}}\},\ \ n<0.
\end{array}
\right.
\]
Define for any $n\in \mathbb{Z}$ the following quantities:%
\begin{equation*}
A_{n}=\int\limits_{U_{n}}V\left( x\right) \left( 1+\left\vert \ln \left\vert
x\right\vert \right\vert \right) dx\ ,\ \ \text{ }\ B_{n}=\Big(
\int\limits_{\left\{ e^{n}<\left\vert x\right\vert <e^{n+1}\right\}
}V^{p}\left( x\right) \left\vert x\right\vert ^{2\left( p-1\right)
}dx\Big) ^{1/p},
\end{equation*}%
where $p>1$ is fixed. Then the following estimate holds%
\begin{equation}
\mathrm{Neg}\left( V\right) \leq 1+C\sum_{\left\{ n\in \mathbb{Z}%
:A_{n}>c\right\} }\sqrt{A_{n}}+C\sum_{\left\{ n\in \mathbb{Z}%
:B_{n}>c\right\} }B_{n},  \label{NegIn}
\end{equation}%
where $C,c$ are positive constants depending only on $p$.

For example, (\ref{NegIn}) implies the finiteness of $\mathrm{Neg}\left(
V\right) $ provided $V$ is locally bounded and 
%\begin{equation*}
$
V\left( x\right) =o\Big( \frac{1}{\left\vert x\right\vert ^{2}\ln
^{2}\left\vert x\right\vert }\Big) \ \ \text{as }x\rightarrow \infty ,
$
%\end{equation*}%
which cannot be seen by any previously known method. 

\end{document}
