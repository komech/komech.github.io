\documentclass[10pt,a4paper]{article}

%% Title/abstract template for the conference in honour of Mark Vishik
%% The original is at http://www.dynamics.iitp.ru/vishik/abstract.tex
%%
%% Instructions:
%% 1. Title and abstract should be in English
%% 2. You are welcome to include the Russian translation (optional).
%%    Russian abstract could be shorter than English.
%% 3. Please make sure that your input, including References, is on one page

%% Enabling Cyrillic input with koi8-r encoding:
\usepackage[T2A]{fontenc}
\usepackage[koi8-r]{inputenc}
\usepackage[russian,english]{babel}

%% Standard packages and definitions:
\usepackage{amssymb}
\usepackage{latexsym}
\usepackage{amsmath}
\def\R{\mathbb{R}}
\def\C{\mathbb{C}}
\def\Z{\mathbb{Z}}
\def\Q{\mathbb{Q}}
\pagestyle{empty}

\begin{document}
\begin{center}

%% Title in English:
{\Large Classical solutions of the Vlasov--Poisson equations in a half-space}

\bigskip

%% Author 1:
{\sc A.L. Skubachevskii}

%% Address:
%{\small\it Peoples' Friendship University of Russia, 117198, Moscow, Miklukho-Maklaya str. 6, Russia}
{\small\it Peoples' Friendship University of Russia, Moscow 117198, Russia}

%% Email:
{\small\rm skub@lector.ru}

\end{center}

\bigskip

%% Abstract in English:
We consider the Vlasov system of equations describing the evolution
of distribution functions of the density for the charged particles
in a rarefied plasma. We study the Vlasov system in
$\R_+^3\times\R^3$ with initial conditions for distribution
functions $f^\beta\big|_{t=0}=f_0^\beta(x,p)$, $\beta=\pm1$, and the
Dirichlet or Neumann boundary conditions for the potential of an
electric field for $x_1=0$, where $f_0^\beta(x,p)$ is the initial
distribution function (for positively charged ions if $\beta=+1$ and
for electrons if $\beta=-1$) at the point~$x$ with impulse~$p$,
$\R_+^3=\{x\in\R^3\colon x_1>0\}$. Assume that initial distribution
functions are sufficiently smooth and ${\rm supp}
f_0^\beta\subset(\R_\delta^3 \cap B_\lambda(0))\times B_\rho(0)$,
$\delta,\lambda,\rho>0$, and the magnetic field $H(x)$ is also
sufficiently smooth and has a~special structure near the boundary
$x_1=0$, where $\R_\delta^3=\{x\in\R^3\colon x_1>\delta\}$. Then we
prove that for any $T>0$ there is a~unique classical solution of the
Vlasov system in $\R_+^3\times\R^3$ for $0<t<T$ if
$\|f_0^\beta\|<\varepsilon$, where
$\varepsilon=\varepsilon(T,\delta,\rho,\|H\|)$ is sufficiently
small.

This work was supported by the RFBR (grant No.\,10-01-00395).


\end{document}
