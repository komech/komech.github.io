\documentclass[10pt,a4paper]{article}

%% Title/abstract template for the conference in honour of Mark Vishik
%% The original is at http://www.dynamics.iitp.ru/vishik/abstract.tex
%%
%% Instructions:
%% 1. Title and abstract should be in English
%% 2. You are welcome to include the Russian translation (optional).
%%    Russian abstract could be shorter than English.
%% 3. Please make sure that your input, including References, is on one page

%% Enabling Cyrillic input with koi8-r encoding:
\usepackage[T2A]{fontenc}
\usepackage[koi8-r]{inputenc}

\usepackage[russian,english]{babel}

%% Standard packages and definitions:
\usepackage{amssymb}
\usepackage{latexsym}
\usepackage{amsmath}
\def\R{\mathbb{R}}
\def\C{\mathbb{C}}
\def\Z{\mathbb{Z}}
\def\Q{\mathbb{Q}}
\pagestyle{empty}

\begin{document}
\begin{center}

%% Title in English:
{\Large Operators with symbolic hierarchies on stratified spaces}

\bigskip

%% Author 1:
{\sc Bert-Wolfgang Schulze}

%% Address:
{\small\it University of Potsdam, Potsdam 14469, Germany}

%% Email:
{\small\rm schulze@math.uni-potsdam.de}

\end{center}

\bigskip

%% Abstract in English:

Manifolds $M$ with higher corners or edges of order  $k \in \mathbb{N}$ are (in our notation) special stratified spaces, where $k=0$ corresponds to smoothness, $k = 1$ to conical or edge singularities, especially smooth boundaries. Manifolds with singularities of order $k$ form a category $ \mathcal{M}_k.$ The stratification $s(M)=(s_0(M),s_1(M),\dots ,s_k(M))$  induces a principal symbolic hierarchy $$\sigma (A)=(\sigma _0(A),\sigma _1(A),\dots ,\sigma _k(A))$$ of operators $A$ over $s_0(M),$ degenerate in a typical way in the representation over the stretched version $\mathbb{M}$ of $M.$ The component $\sigma_0(A)$ is the standard homogeneous principal symbol on the main stratum $s_0(M)$; the component $\sigma_j(A),\,j>0,$ lives on $s_k(M))$ and is operator-valued. The symbolic hierarchy admits notions of ellipticity and the construction of parametrices within suitable algebras of degenerate pseudo-differential operators. We present some new developments in this field which has a long history through achievements of numerous Russian authors and other schools worldwide. Further progress is stimulated by the desire to reach new models of applications, see, for instance, \cite{Flad3}. Moreover, the tower of operator algebras with increasing $k$ still contains many new challenges. The methods of the author have been stimulated very much by the works \cite{Vivs2}, \cite{Vivs3}, \cite{Eski2}, since the case of manifolds with edge contains boundary value problems with and without the transmission property at the boundary.



\begin{thebibliography}{6}

\bibitem{Eski2} G.I. Eskin, \textit{Boundary value problems for elliptic pseudodifferential equations}, Transl. of Nauka, Moskva, 1973, Math. Monographs, Amer. Math. Soc. \textbf{52}, Providence, Rhode Island 1980.

\bibitem{Flad3} H.-J. Flad, G. Harutyunyan, R. Schneider, and B.-W. Schulze, \textit{Explicit Green operators for quantum mechanical Hamiltonians.I. The hydrogen atom}, arXiv:1003.3150v1 [math.AP], 2010. manuscripta math. \textbf{135}(2011), 497-519.

%\bibitem{Haru13} G. Harutjunjan and B.-W. Schulze, \textit{Elliptic mixed, transmission and singular crack problems}, European Mathematical Soc., Z\" urich, 2008.

%\bibitem{Kond1} V.A. Kondratyev, \textit{Boundary value problems for elliptic equations in domains with conical points}, Trudy Mosk. Mat. Obshch. \textbf{16} (1967), 209-292.

\bibitem{Schu75} B.-W. Schulze \textit{The iterative structure of the corner calculus}, Oper. Theory: Adv. Appl. \textbf{213}, Pseudo-Differential Operators: Analysis, Application and Computations (L. Rodino et al. eds.), Birkh\" auser Verlag, Basel, 2011, pp. 79-103.

\bibitem{Vivs2} M.I. Vishik and G.I. Eskin, \textit{Convolution equations in a bounded region}, Uspekhi Mat. Nauk \textbf{20}, 3 (1965), 89-152.

\bibitem{Vivs3} M.I. Vishik and G.I. Eskin, \textit{Convolution equations in bounded domains in spaces with weighted norms}, Mat. Sb. \textbf{69}, 1 (1966),65-110.

%\bibitem[8]{Vivs4} M.I. Vishik and V.V. Grushin, \textit{On a class of degenerate elliptic equations of higher orders}, Mat. Sb. \textbf{79}, 1 (1969), 3-36.\\\end{thebibliography}

\end{thebibliography}

\end{document}
