\documentclass[10pt,a4paper]{article}

%% Title/abstract template for the conference in honour of Mark Vishik
%% The original is at http://www.dynamics.iitp.ru/vishik/abstract.tex
%%
%% Instructions:
%% 1. Title and abstract should be in English
%% 2. You are welcome to include the Russian translation (optional).
%%    Russian abstract could be shorter than English.
%% 3. Please make sure that your input, including References, is on one page

%% Enabling Cyrillic input with koi8-r encoding:
\usepackage[T1]{fontenc}
\usepackage[latin1]{inputenc}
\usepackage[english]{babel}
%\usepackage[T2A]{fontenc}
%\usepackage[koi8-r]{inputenc}
%\usepackage[russian,english]{babel}

%% Standard packages and definitions:
\usepackage{amssymb}
\usepackage{latexsym}
\usepackage{amsmath}
\def\R{\mathbb{R}}
\def\C{\mathbb{C}}
\def\Z{\mathbb{Z}}
\def\Q{\mathbb{Q}}
\pagestyle{empty}

\begin{document}
\begin{center}

%% Title in English:
{\Large Critical manifold in the space of contours in Stokes-Leibenson problem for Hele-Shaw flow}

\bigskip

%% Author 1:
{\sc A.S. Demidov}

%% Address:
{\small\it Moscow State University, Moscow 119992, Russia}

%% Email:
{\small\rm alexandre.demidov@mtu-net.ru}

\bigskip

%% Author 2:
{\sc J.-P. Loh\'eac}

{\small\it \'Ecole centrale de Lyon, Institut Camille-Jordan}

{\small\rm Jean-Pierre.Loheac@ec-lyon.fr}

\bigskip

%% Author 3:
{\sc V. Runge}

{\small\it \'Ecole centrale de Lyon, Institut Camille-Jordan}

{\small\rm Vincent.Runge@ec-lyon.fr}

\end{center}

\bigskip

%% Abstract in English:

We here deal with the Stokes-Leibenson problem for a
punctual Hele-Shaw flow. By using a geometrical transformation
inspired by Helmholtz-Kirchhoff method, we introduce an
integro-differential problem which leads to the construction of
a discrete model.
We first give a short recall about the source-case: global in time
existence and uniqueness result for an initial contour close to a
circular one, investigation of the evolutionary structure of the solution.
Our main subject concerns the development of a numerical model
in order to get some qualitative properties of the motion.
This model provides numerical experiments which confirm the
existence of a critical manifold of codimension 1 in some space
of contours.
This manifold contains one attractive point in the source-case
corresponding to a circular contour centered at the source-point.
In the sink-case, every point of this manifold seems to be attractive.
In particular, we present some numerical experiments linked to
fingering effects.

\begin{thebibliography}{7}

\bibitem{Ga}
{L.A. Galin,}
\emph{Unsteady filtration with a free surface}.
Dokl. Akad. Nauk SSSR {\bf 47} (1945), 250--253.

\bibitem{Le}
{L.S. Leibenson,}
\emph{Oil producing mechanics, Part II}.
Moscow, Neftizdat,  1934.

\bibitem{OH}
{J.R. Ockendon, S.D. Howison,}
\emph{Kochina and Hele-Shaw in
modern mathematics, natural science and industry}.
J. Appl. Math. Mech. {\bf 66} (2002), No~3, 505--512.

\bibitem{P1}
{Y.Ya. Polubarinova-Kochina,}
\emph{On the motion of the oil contour}.
{Dokl. Akad. Nauk SSSR} {\bf 47} (1945), 254--257.

\bibitem{P2}
{P.Ya. Polubarinova-Kochina,}
\emph{Concerning unsteady motions in the theory of filtration}.
{Prik. Mat. Mech.} {\bf 9} (1945), 79--90.

\bibitem{St}
{G.G. Stokes,}
\emph{Mathematical proof of the identily of the
stream-lines obtained by means of viscous film with those of a
perfect fluid moving in two dimensions}.
{Brit. Ass. Rep.} \textbf{143} (1898) (Papers, V, 278).

\end{thebibliography}

\end{document}
