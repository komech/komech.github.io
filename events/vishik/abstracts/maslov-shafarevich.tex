\documentclass[10pt,a4paper]{article}

%% Title/abstract template for the conference in honour of Mark Vishik
%% The original is at http://www.dynamics.iitp.ru/vishik/abstract.tex
%%
%% Instructions:
%% 1. Title and abstract should be in English
%% 2. You are welcome to include the Russian translation (optional).
%%    Russian abstract could be shorter than English.
%% 3. Please make sure that your input, including References, is on one page

%% Enabling Cyrillic input with koi8-r encoding:
%\usepackage[T2A]{fontenc}
%\usepackage[koi8-r]{inputenc}
%\usepackage[russian,english]{babel}
%\usepackage[english]{babel}

%% Standard packages and definitions:
\usepackage{amssymb}
\usepackage{latexsym}
\usepackage{amsmath}
\def\R{\mathbb{R}}
\def\C{\mathbb{C}}
\def\Z{\mathbb{Z}}
\def\Q{\mathbb{Q}}
\def\p{\partial}
\pagestyle{empty}

\begin{document}
\begin{center}

%% Title in English:
{\Large Asymptotic solutions of the Navier-Stokes equations and scenario of turbulence development}

\bigskip

%% Author 1:
{\sc Victor Maslov}

%% Address:
{\small\it Moscow State University, Moscow 119991, Russia}

%% Email:
{\small\rm v.p.maslov@mail.ru}

\bigskip

%% Author 2:
{\sc Andrei Shafarevich}

%% Address:
{\small\it Moscow State University, Moscow 119991, Russia}

%% Email:
{\small\rm shafarev@yahoo.com}

\end{center}

\bigskip


We discuss asymptotic solutions of the Navier-Stokes equations,
describing periodic collections of vortices in 3D space.
These solutions are connected with topological invariants
of divergence-free vector fields.
Equations, describing evolution of vortices,
are defined on a graph -- Reeb graph of the stream function
or Fomenko molecule of the Liouville foliation.
Homogenization with respect to
the periodic structure leads to equations
coinciding with Reynolds equations.
It is well known that existence of the Reynolds stresses leads
to the growth of the energy and entropy of the fluid.
As the entropy reaches certain critical value,
the molecules of the fluid have to form
``clusters'' which leads to the occurrence of turbulence.


\end{document}
